\documentclass[xelatex]{beamer}
\usetheme[titleformat title=smallcaps, titleformat section=smallcaps, titleformat frame=smallcaps]{metropolis}

%%% Load packages %%%%%%%%%%%%%%%%%%%%%%%%%%%%%%%%%%%%%%%%%%%%%%%%

% \usepackage{amsfonts}
% \usepackage{amssymb}
% \usepackage{amsthm}
% \usepackage{amsopn} 

\usepackage{fontspec}
%[
%Renderer=Graphite,
%RawFeature={onum}]
\setmainfont{Libertinus Serif}[
Ligatures={Common,TeX},
Numbers={OldStyle, Proportional}
]
\setsansfont{Libertinus Sans}[
  Ligatures={Common,TeX},
  Numbers={OldStyle, Proportional}
  ]
\setmonofont{Libertinus Mono}

\usepackage{amsmath}
\usepackage{unicode-math}
\setmathfont[Scale=0.85]{Libertinus Math}

% \usefonttheme{professionalfonts} % required for mathspec
% \usepackage{mathspec}
% \setmathsfont(Digits,Latin)[Numbers={Lining, Proportional},Scale=MatchLowercase]{Libertinus Math}
% 


% Use for large room or with an underpowered projector
% \setsansfont[BoldFont={Fira Sans SemiBold}]{Fira Sans Book}

%%% hyperref setup %%%%%%%%%%%%%%%%%%%%%%%%%%%%%%%%%%%%%%%%%%%%%%%%
\hypersetup{pdfpagemode=FullScreen,pdffitwindow=true,pdfpagelayout=SinglePage}

%%% metropolis config %%%%%%%%%%%%%%%%%%%%%%%%%%%%%%%%%%%%%%%%%%%%%%%%
% \metroset

%%% Title and authors information %%%%%%%%%%%%%%%%%%%%%%%%%%%%%%%%%%%%%%%%%%%%%%%%
\title[Crisis] % (optional, only for long titles)
{A Fog Computing Prototype}
\subtitle{Course Project for Big Data Analytics --- Winter 2019}
\author[Ali, Marco] % (optional, for multiple authors)
{Ali Alizadeh Mansouri \and Marco Sassano}
\institute[Concordia University] % (optional)
{
    Concordia University\\
    %   University Here
    %   \and
    %   Institute of Theoretical Philosophy\\
    %   University There
}
\date[April 15, 2019] % (optional)
{April 15, 2019}
\subject{Course Project for Big Data Analytics --- Winter 2019}
    
%%% TOC at each subsection %%%%%%%%%%%%%%%%%%%%%%%%%%%%%%%%%%%%%%%%%%%%%%%%
% \AtBeginSubsection[]
% {
%   \begin{frame}
%     \frametitle{Table of Contents}
%     \tableofcontents[currentsection,currentsubsection]
%   \end{frame}
% }


%%%%%%%%%%%%%%%%%%%%%%%%%%%%%%%%%%%%%%%%%%%%%%%%%%%
%%% begin document %%%%%%%%%%%%%%%%%%%%%%%%%%%%%%%%%%%%%%%%%%%%%%%%

\begin{document}
\frame{\titlepage}

%%% TOC %%%%%%%%%%%%%%%%%%%%%%%%%%%%%%%%%%%%%%%%%%%%%%%%

\begin{frame}
    \frametitle{Table of Contents}
    \tableofcontents[currentsection]
\end{frame}

% \frame{\sectionpage}
\section[Section]{My section}
\subsection[Subsection]{My subsection}

\subsubsection[Subsubsection]{My subsubsection}


  \begin{frame}
    \frametitle{This is the first slide}
    %Content goes here
  \end{frame}

  \begin{frame}
    \frametitle{This is the second slide}
    \framesubtitle{A bit more information about this}

    \begin{itemize}[<+->]
      \item The truths of arithmetic which are independent of PA in some 
      sense themselves `{contain} essentially {\color{blue}{hidden higher-order}},
       or infinitary, concepts'???
      \item `Truths in the language of arithmetic which \ldots
      \item	That suggests stronger version of Isaacson's thesis. 
      \end{itemize}

    \begin{equation*}
      \symup{e} = \lim_{n\to \infty} \left(1 + \frac{1}{n}\right)^n
    \end{equation*}
    % \pause
    \begin{equation*}
      s_{t}=\begin{cases}
      \bar{s}, & t\in \left\lbrace 0,\dots, T-1\right\rbrace  \\
      \tilde{s}, & t\geq T Q125\% Question \alpha \gamma \varGamma
      \end{cases}
    \end{equation*} 
    %More content goes here
  \end{frame}
% etc
\end{document}